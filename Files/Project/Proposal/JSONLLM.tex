% !TeX root = JSONLLM.tex

% !TeX root = JSONLLM.tex

\documentclass{article}

% \usepackage{neurips_2024} % ready for submission
\usepackage[preprint]{neurips_2024} % to compile a preprint version, e.g., for submission to arXiv, add the [preprint] option
% \usepackage[final]{neurips_2024} % to compile a camera-ready version, add the [final] option
% \usepackage[nonatbib]{neurips_2024} % to avoid loading the natbib package, add option nonatbib

% === INÍCIO DAS ANOTAÇÕES ADVINDAS DO MODELO NEURIPS_2024 === %

% \PassOptionsToPackage{options}{natbib} % If you wish to load the \verb+natbib+ package with options, you may add the following before loading the \verb+neurips_2024+ package:
% \citet{fulano_2018} afirmam que \dots
% \footnote{Sample of the first footnote.}
% Caption after the figure
% Caption before table
% \begin{ack} % Agradecimento aos institutos financiadores e/ou empresas envolvidas no projeto. <https://neurips.cc/Conferences/2024/PaperInformation/FundingDisclosure>
% \end{ack}
% referências podem ser \small (9 pts); qualquer estilo de citação, desde que consistente.
% \appendix % Optionally include supplemental material (complete proofs, additional experiments and plots) in appendix. All such materials \textbf{SHOULD be included in the main submission.}

% === FIM DAS ANOTAÇÕES ADVINDAS DO MODELO NEURIPS_2024 === %

\usepackage[utf8]{inputenc} % allow utf-8 input
\usepackage[T1]{fontenc}    % use 8-bit T1 fonts
\usepackage{hyperref}       % hyperlinks
\usepackage{url}            % simple URL typesetting
\usepackage{booktabs}       % professional-quality tables
\usepackage{amsfonts}       % blackboard math symbols
\usepackage{nicefrac}       % compact symbols for 1/2, etc.
\usepackage{microtype}      % microtypography
\usepackage{xcolor}         % colors
\usepackage{comment}        % enables the comment environment
\usepackage[inline]{enumitem} % inline enumeration

\title{Project Proposal\\JSONLLM: Extração Escalável de Atributos de Produtos com Grandes Modelos de Linguagem (LLMs)}

\author{
    João Vítor Fernandes Dias
    % \thanks{ORCID: \href{https://orcid.org/0000-0002-8156-9551}{0000-0002-8156-9551}}
    \thanks{
        ORCID: \href{https://orcid.org/0000-0002-8156-9551}{0000-0002-8156-9551};
        GitHub: \href{https://github.com/jvfd3}{jvfd3};
        LinkedIn: \href{https://www.linkedin.com/in/jvfd3}{jvfd3}
    } \\
    Computer Science Postgraduate Program\\
    2024711370\\
    Federal University of Minas Gerais\\
    Belo Horizonte, MG; 31270-901 \\
    \texttt{joaovitorfd2000@gmail.com} \\
    \And % Quebra de linha automática
    % \AND % Quebra de linha forçada
    Melissa Dias Mattos \\
    Computer Science Postgraduate Program\\
    2024675063\\
    Federal University of Minas Gerais\\
    Belo Horizonte, MG; 31270-901 \\
    \texttt{melissadmattos366@gmail.com} \\
}

\begin{comment}
\appendix
\section{Appendix / supplemental material}
\newpage
\section*{NeurIPS Paper Checklist}
\end{comment}


\begin{document}

\maketitle

\begin{abstract} % 150-250 words; problem, solution, objectives, expected outcomes.
    % Problem
    Muitos produtos de \textit{e-commerce} possuem múltiplos atributos -- como tamanho, cor e material -- que são essenciais para auxiliar os clientes em tomadas de decisões de compra informadas. No entanto, a extração e a organização dessas informações a partir de descrições textuais de produtos não estruturadas é uma tarefa desafiadora.
    % Solution
    Neste trabalho, propomos o JSONLLM, uma abordagem que explora o potencial dos grandes modelos de linguagem (LLMs) para identificar e estruturar atributos de produtos em formato JSON. Nosso método utiliza as capacidades avançadas de compreensão semântica dos LLMs para reconhecer e categorizar com precisão informações provenientes de descrições diversas e complexas.
    % Objectives
    Os objetivos principais deste projeto incluem: integrar LLMs para a extração de atributos de produtos, garantindo a escalabilidade com sistemas reais de \textit{e-commerce}, padronizando e avaliando a precisão do modelo proposto em comparação a técnicas de \textit{prompt engineering}.
    % Expected outcomes
    Espera-se que o sistema resultante melhore significativamente a precisão da extração de atributos de produtos, facilitando a criação de catálogos de produtos mais ricos e detalhados. Vale ressaltar que a estruturação em JSON facilita a integração com sistemas de \textit{e-commerce} existentes, melhorando a experiência do usuário final.
\end{abstract}

\section{Introdução} % Introduction and Motivation: Background, problem statement, significance, motivation.

A extração e estruturação de atributos de produtos em \textit{e-commerce} representa um desafio crucial devido à diversidade e complexidade das descrições de produtos não estruturadas. Muitos produtos possuem múltiplos atributos essenciais, como tamanho, cor e material, que são fundamentais para decisões informadas de compra. Métodos tradicionais de extração de informações frequentemente enfrentam limitações em precisão e adaptabilidade a descrições variadas \citet{ExtractGPT2023}. Recentemente, o uso de Grandes Modelos de Linguagem (LLMs) tem demonstrado potencial significativo para superar esses desafios, permitindo uma extração mais precisa e estruturada em formatos como JSON, facilitando a integração com sistemas de \textit{e-commerce} existentes \citet{PAE2024}. Esta motivação conduz à proposta do sistema JSONLLM, que visa explorar as capacidades avançadas dos LLMs para melhorar a qualidade da extração e apresentação de atributos de produtos.

\section{Objetivos} % Project Objectives: SMART objectives (clear, specific, measurable, achievable, relevant, and time-bound).

% (1) desenvolver um pipeline que integra LLMs para a extração de atributos de produtos; (2) definir um esquema padronizado para a representação dos atributos extraídos; (3) avaliar a precisão e eficiência do modelo proposto em comparação a técnicas de \textit{prompt engineering}; e (4)

Os principais objetivos deste projeto são:
\begin{enumerate*}
    \item Desenvolver um pipeline que integre LLMs para a extração automática e precisa de atributos de produtos a partir de descrições não estruturadas;
    \item Criar um esquema padronizado em formato JSON para representar os atributos extraídos de forma clara e interoperável;
    \item Avaliar a precisão e eficiência do modelo proposto em comparação com métodos tradicionais de extração, mediante experimentos controlados;
    \item Garantir a escalabilidade e aplicabilidade do sistema em ambientes reais de \textit{e-commerce}, com potencial para integração fácil.
\end{enumerate*}

\section{Metodologia} % Proposed Methodology: Data collection/preparation, model/architecture selection, implementation details, experimental design, evaluation plan.

\urldef\maveurl\url{https://github.com/google-research-datasets/MAVE}
\urldef\trecurl\url{https://huggingface.co/datasets/trec-product-search/product-recommendation-2025}

A metodologia envolve a coleta e preparação de um conjunto de dados representativo de descrições de produtos do \textit{e-commerce}, apoiada por bases públicas como o dataset MAVE (\textit{Multi-source Attribute Value Extraction}), que possui milhões de anotações de pares atributo-valor \citet{MAVE2024}\footnote{\maveurl}. Após o treino com as bases de dados anotadas o modelo será aplicado no \textit{dataset} da \textit{Track Product Search} da TREC (\textit{Text REtrieval Conference})\footnote{\trecurl} \citet{TRECProductSearch_2023} para avaliação empírica do método de enriquecimento dos dados.

Será adotada uma arquitetura baseada em LLMs, com experimentações utilizando modelos \textit{open-source}, explorando técnicas de \textit{few-shot} e \textit{zero-shot learning} para extração estruturada. Em etapa posterior, através do \textit{finetuning} um modelo pré-treinado será aprimorado através dos exemplos anotados e sua performance será avaliada.

O \textit{pipeline} incluirá a padronização dos atributos extraídos em JSON, com validações automatizadas e refinamento iterativo via técnicas de autoaperfeiçoamento (\textit{self-refinement}) para aumentar a precisão \citet{SelfRefinementLLM2024}. O desempenho do sistema será avaliado por métricas de precisão, \textit{recall} e f1-\textit{score}, comparado a abordagens baseadas em aprendizado de máquina tradicional e ferramentas existentes.

\section{Resultados Esperados e Contribuições} % Expected Outcomes and Contributions: Anticipated results, contributions to LLM field, potential applications.

Espera-se que o sistema JSONLLM melhore significativamente a precisão da extração de atributos de produtos, permitindo a geração de catálogos detalhados e padronizados. Assim os catálogos poderão ser dispostos em interfaces agradáveis aos usuários para facilitar a avaliação de opções disponíveis.

A contribuição do projeto inclui o desenvolvimento de um \textit{pipeline} LLM para extração de atributos, o avanço no uso de JSON como padrão para dados estruturados em contexto comercial e a validação empírica da superioridade do método em relação a abordagens tradicionais \citet{ExtractGPT2023}. Além disso, o projeto pode ser aplicado em múltiplos domínios de comércio eletrônico, bem como em outras bases de dados, potencializando a interoperabilidade entre sistemas diversos.

% \newpage

\section{Cronograma} % Timeline and Milestones: 1. Literature Review; 2. Data Preparation; 3. Model Implementation; 4. Experimentation; 5. Report Draft; 6. Final Submission.

\begin{table}[htbp] \centering
    \caption{Cronograma do Projeto} \label{tab:cronograma}
    \begin{tabular}{cl}
        \toprule
        \textbf{Prazo} & \textbf{Atividade}                                    \\
        \midrule
        02/10          & Proposta do projeto                                   \\
        09/10          & Revisão de literatura                                 \\
        13/10          & Preparação dos dados                                  \\
        14/10          & \textit{Mid-Project Check-in} (\textit{Report Draft}) \\
        21/10          & Implementação do modelo                               \\
        28/10          & Experimentos e avaliações                             \\
        18/11          & Submissão Final                                       \\
        25/11          & Apresentações                                         \\
        \bottomrule
    \end{tabular}
\end{table}

% \section{Referências} % References: Cite relevant literature with consistent formatting.

\bibliographystyle{apalike}
\bibliography{referencias}

\appendix

\section{Desafios e Estratégias de Mitigação} % Potential Challenges and Mitigation Strategies: Potential challenges and mitigation strategies.

Entre os principais desafios destacam-se a variabilidade e ambiguidade nas descrições dos produtos que dificultam a extração consistente de atributos; a necessidade de adaptação do modelo a múltiplos domínios de produto; e a escalabilidade do \textit{pipeline} para grandes volumes de dados. Para mitigar esses desafios, serão aplicadas estratégias de refinamento iterativo para corrigir erros, adoção de esquemas JSON flexíveis e extensíveis, e uso de conjuntos de dados amplos e diversificados para o treinamento e validação \citet{SelfRefinementLLM2024}.

\end{document}
